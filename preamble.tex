\usepackage[utf8]{inputenc}
\usepackage[T1]{fontenc}
\usepackage[pdftex]{graphicx}
\newcommand{\HRule}{\rule{\linewidth}{0.5mm}}
\newcommand{\thline}{\rule{}{0.5mm}}
\usepackage{parskip}
\usepackage{color}
\definecolor{mygreen}{RGB}{28,172,0} % color values Red, Green, Blue
\definecolor{mylilas}{RGB}{170,55,241}
\usepackage[usenames,dvipsnames]{xcolor}
\usepackage{graphicx}
\usepackage{amsmath}
\usepackage{fancyvrb}
\usepackage{booktabs}
\usepackage{color}
\definecolor{myblue}{rgb}{.8, .8, 1}
\usepackage{float}
\usepackage[caption = false]{subfig}
\usepackage{amsmath}
\usepackage{empheq}
\usepackage{times}
\usepackage[scaled=0.92]{helvet}
\usepackage{courier}
\usepackage[left=3cm,right=2.5cm,top=2.5cm,bottom=2.5cm]{geometry}
\usepackage{gensymb}
\usepackage{physics}

\newlength\mytemplen
\newsavebox\mytempbox
\usepackage{epstopdf}
\makeatletter
\newcommand\mybluebox{%
    \@ifnextchar[%]
       {\@mybluebox}%
       {\@mybluebox[0pt]}}

\def\@mybluebox[#1]{%
    \@ifnextchar[%]
       {\@@mybluebox[#1]}%
       {\@@mybluebox[#1][0pt]}}

\def\@@mybluebox[#1][#2]#3{
    \sbox\mytempbox{#3}%
    \mytemplen\ht\mytempbox
    \advance\mytemplen #1\relax
    \ht\mytempbox\mytemplen
    \mytemplen\dp\mytempbox
    \advance\mytemplen #2\relax
    \dp\mytempbox\mytemplen
    \colorbox{myblue}{\hspace{1em}\usebox{\mytempbox}\hspace{1em}}}

\makeatother

\usepackage[pdfpagemode={UseOutlines},bookmarks=true,bookmarksopen=true,
   bookmarksopenlevel=0,bookmarksnumbered=true,hypertexnames=false,
   colorlinks,linkcolor={black},citecolor={black},urlcolor={black},
   pdfstartview={FitV},unicode,breaklinks=true]{hyperref}
\pdfstringdefDisableCommands{
   \let\\\space
}

\usepackage{fancyhdr}

\pagestyle{fancy}
\renewcommand{\sectionmark}[1]{\markboth{#1}{}} % set the \leftmark

\fancyhf{}
\fancyhead[R]{} % predefined ()
\fancyhead[L]{\leftmark} % 1. sectionname
\fancyfoot[C]{\thepage}
\fancypagestyle{mine}{%
  \fancyhf{}%
  \renewcommand{\headrulewidth}{0pt}%
}

\makeatletter
\newcommand{\verbatimfont}[1]{\def\verbatim@font{#1}}%
\makeatother

\usepackage{listings}
\renewcommand\lstlistlistingname{List of Codes}
\renewcommand{\lstlistingname}{Code}

\lstset{%
  language = Octave,
  backgroundcolor=\color{white},   
  basicstyle=\footnotesize\ttfamily,       
  breakatwhitespace=false,         
  breaklines=true,                 
  captionpos=b,                   
  commentstyle=\color{gray},    
  deletekeywords={...},           
  escapeinside={\%*}{*)},          
  extendedchars=true,              
  frame=single,                    
  keepspaces=true,                 
  keywordstyle=\color{orange},       
  morekeywords={*,...},            
  numbers=left,                    
  numbersep=5pt,                   
  numberstyle=\footnotesize\color{gray}, 
  rulecolor=\color{black},         
  rulesepcolor=\color{blue},
  showspaces=false,                
  showstringspaces=false,          
  showtabs=false,                  
  stepnumber=2,                    
  stringstyle=\color{Apricot},    
  tabsize=2,                       
  title=\lstname,
  emphstyle=\bfseries\color{CornflowerBlue}%  style for emph={} 
} 

%% language specific settings:
\lstdefinestyle{Arduino}{%
    language = Octave,
    keywords={void, int boolean, digitalWrite, digitalRead, delay, millis, analogRead, print, available, begin, end, flish, println, read, Serial, pinMode, SoftwareSerial},%                 define keywords
    morecomment=[l]{//},%             treat // as comments
    morecomment=[s]{/*}{*/},%         define /* ... */ comments
    emph={HIGH, OUTPUT, LOW}%        keywords to emphasize
}

\lstset{language=Matlab,%
    %basicstyle=\color{red},
    breaklines=true,%
    morekeywords={matlab2tikz},
    keywordstyle=\color{blue},%
    morekeywords=[2]{1}, keywordstyle=[2]{\color{black}},
    identifierstyle=\color{black},%
    stringstyle=\color{DarkOrchid},
    commentstyle=\color{OliveGreen},%
    showstringspaces=false,%without this there will be a symbol in the places where there is a space
    numbers=left,%
    numberstyle={\tiny \color{black}},% size of the numbers
    numbersep=9pt, % this defines how far the numbers are from the text
    emph=[1]{for,end,break, if, else, while},emphstyle=[1]\color{red}, %some words to emphasise
    %emph=[2]{word1,word2}, emphstyle=[2]{style},    
}
\lstdefinestyle{Matlab}{language = Matlab}
